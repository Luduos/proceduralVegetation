\chapter{Einleitung und Problemstellung}
test
\section{Prozedurale Generierung}

Insbesondere Vegetation - warum, was sind die Schwierigkeiten, wenn man es nicht prozedural generiert.

\section{Bisherige Arbeiten}

test

\section{Ansatz}
Verwendete Ansätze: L-Systeme mit Turtle-Graphik und Space Colonization Algorithmus. Kleine Einleitung zu beiden Ansätzen, bisherige Verwendung.

Beschreibung - was wird mit Baumstruktur / Baum-Graph gemeint?

Um eine Verbindung zwischen der visuellen Repräsentation und der Implementierung zu schaffen, werden an manchen Stellen die Begriffe \glqq Ast\grqq beziehungsweise \glqq Zweig\grqq anstatt \glqq Kante\grqq oder \glqq Knoten\grqq verwendet. Ein Astsegment entspricht einer Kante im graphentheoretischen Baum. Eine Abzweigung oder Verzweigung entspricht den Kanten eines Knotens zu mehreren seiner Nachfolger.

\section{Unreal Engine 4}

Allgemeine Erklärung - was ist die Unreal Engine 4, was stellt sie zur Verfügung, wie entwickelt man dafür? Insbesondere Eingabe von Parametern über den Editor --> wichtig für Implementierung.

Erklärung - was ist ein Unreal-Akteur? Was sind die Komponenten, die diesem hinzugefügt werden können?
Sämtliche Visualisierungen von L-Systemen wurden mithilfe des Unreal Engine 4 Projekts erstellt.

Erklärung - was ist ein L-System-Akteur, was ist ein Space Colonization-Akteur

Alle Längenangaben entsprechen der in der Unreal Engine 4 verwendeten Längeneinheit $cm$.