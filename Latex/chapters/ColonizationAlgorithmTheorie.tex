\chapter{Space Colonization Algorithmus}

TODO: Zusammenfassung Space Colonization Algorithm.

\section{Ursprung}
Der Space Colonization Algorithmus wurde ursprünglich zur Modellierung und Visualisierung von Blattvenen vorgeschlagen und basiert auf der Wirkung des Pflanzenhormons Auxin. Dieser Hormonstoff entsteht im Blatt und wird von bereits existierenden Blattvenen angezogen, der resultierende Hormonstrom führt zur Bildung von neuen Venen im Blatt. Die Simulation dieses Vorgangs führt zu realitätsnahen Venenmustern. \cite[Abschn. 2.5]{LeafVenation:05}

Mithilfe einer Erweiterung in den dreidimensionalen Raum und Nachbearbeitung der Resultate kann eine Vielfalt von Baum- und Strauchstrukturen generiert werden. \cite[Abschn. 1]{SpaceColonizationAlgorithm:07}

\section{Aufbau}

Der Algorithmus verarbeitet eine Menge von Einflusspunkten $S$ und baut darauf basierend einen gerichteten Baum-Graphen $G = \langle V,E\rangle$ auf. Folgende Begriffe werden in diesem Zusammenhang verwendet: 

\begin{description}[labelindent]
	\item[\boldmath$V$] Die Menge der Knotenpunkte. Jeder Knotenpunkt entspricht einem Punkt $\overrightarrow{p_v}$ im dreidimensionalen Raum. \cite[S.358]{ThI:14}\\
	
	\item[\boldmath$E$] Die Menge der Kanten, welche die Vorgänger-Nachfolger-Beziehungen zwischen den Knotenpunkten darstellt. Eine Kante $e \in E$ wird als Tupel $e = (v_1, v_2)$ mit $v_1, v_2 \in V$ dargestellt. Jeder Knoten besitzt maximal einen Vorgänger und eine endliche Menge von Nachfolgern.\cite[S.358]{ThI:14}\\
	
	\item[\boldmath$Wurzel$] Eine Wurzel ist ein Knoten $v \in V$ ohne einen Vorgänger. Der Baum besitzt genau eine Wurzel. \cite[S.358]{ThI:14}\\
	
	\item[\boldmath$Grad$] Der Grad eines Knotens ist definiert als die Anzahl seiner Nachfolger. \cite[S.29]{AlgoDat:14}\\
	
	\item[\boldmath$Tiefe$] Die Tiefe eines Knotens ist definiert als die Länge der Folge von Vorgängern, die durchlaufen werden muss, bis die Wurzel erreicht wurde. Die Wurzel besitzt die Tiefe $0$. \cite[S.30]{AlgoDat:14}
	
\end{description}
Der Algorithmus benötigt die folgenden Eingaben:

\begin{description}[labelindent]
	\item[\boldmath$d_i$] Der Einflussradius. Einflusspunkte prägen den Aufbau des Baums nur, wenn sich Knotenpunkte innerhalb dieses Radius befinden.\\
	
	\item[\boldmath$d_k$] Der Minimalradius. Befindet sich ein Knotenpunkt innerhalb des Minimalradius um einen Einflusspunkt, wird dieser aus der Menge der Einflusspunkte $S$ entfernt.\\
	
	\item[\boldmath$D$] Die Schrittweite. Jeder neu generierte Knotenpunkt wird in diesem Abstand zu seinem Vorgänger positioniert.

\end{description}
\cite[Abschn. 2]{SpaceColonizationAlgorithm:07}
\section{Ablauf}
\label{sec:SCA_Ablauf}
Zu Beginn des Algorithmus werden $N$ Einflusspunkte in einem vorgegebenen Bereich generiert. Dieser Einflussbereich signalisiert die Verfügbarkeit von Raum, in dem der Baum wachsen kann.  \cite[Abschn. 2]{SpaceColonizationAlgorithm:07}

Daraufhin wird der Baum iterativ aufgebaut und durchläuft in jeder Iteration die folgenden Schritte: 

\begin{description}[labelindent]
	\item[\boldmath$1.$] Für jeden Einflusspunkt in $S$ wird der am nächsten liegende Knotenpunkt $v\in V$ bestimmt. Befindet sich $v$ innerhalb des Einflussradius $d_i$ um den Einflusspunkt herum, wird dieser einer zugeordneten Menge $S(v)$ hinzugefügt. $S(v)$ beinhaltet somit alle Einflusspunkte, die einen Einfluss auf den Knotenpunkt ausüben. \cite[Abschn. 2]{SpaceColonizationAlgorithm:07} \label{alg:SCA_1}\\
	
	\item[\boldmath$2.$] Befinden sich Elemente in $S(v)$, wird ein neuer Knotenpunkt $v'$ den Nachfolgern von $v$ hinzugefügt und $v$ als Vorgänger von $v'$ eingetragen.  Alle Punkte in $S(v)$ beeinflussen $v'$ in gleichem Maße, die neue Position $\overrightarrow{p_{v'}}$ des Knotenpunkts berechnet sich somit wie folgt:
	
	\begin{equation}
	\begin{array}{ll}
	\overrightarrow{p_{v'}} & = \overrightarrow{p_v} + D * \dfrac{\overrightarrow{n} }{\lVert\overrightarrow{n}\rVert} \text{  mit  } \overrightarrow{n} = \sum\limits_{s \in S(v)}\dfrac{\overrightarrow{p_s} - \overrightarrow{p_v}}{\lVert \overrightarrow{p_s} - \overrightarrow{p_v} \rVert}
	\end{array}
	\end{equation} 
	
	\cite[Abschn. 2]{SpaceColonizationAlgorithm:07} \label{alg:SCA_2}\\
	
	\item[\boldmath$3.$] Für jeden Einflusspunkt wird überprüft, ob sich ein Knotenpunkt innerhalb des Minimalradius $d_k$ befindet. Existiert ein solcher Knotenpunkt, wird der Einflusspunkt aus der Menge der Einflusspunkte $S$ entfernt. \cite[Abschn. 2]{SpaceColonizationAlgorithm:07} \label{alg:SCA_3}
\end{description}

TODO: Ablauf des Algorithmus. als Bilder.


\section{Modellierung von Baumstrukturen}

Der Space Colonization Algorithmus liefert einen Baum-Graphen, der Knotenpunkte enthält, welche Positionen im dreidimensionalen Raum darstellen. Um diese Knotenpunkte in Form von baumähnlichen Strukturen zu visualisieren, wird die Prozedur erweitert. Der gesamte Generierung der Baumstrukturen läuft nun wie folgt ab:

\begin{description}[labelindent]
	\item[\boldmath$1.$] Der Einflussbereich wird mit der vorgegeben Anzahl von Einflusspunkten gefüllt. 
	TODO: Bild
	\\
	
	\item[\boldmath$2.$] Der Baum-Graph wird wie in Abschnitt \ref{sec:SCA_Ablauf} beschrieben iterativ generiert.
	TODO: Bild
	\\
	
	\item[\boldmath$3.$] Die Nachfolger jedes Knotenpunkts werden einander angenähert, um eine Verringerung der Abzweigungswinkel zwischen den verbindenden Kanten zu erreichen. Dies führt zu einer insgesamt geschmeidigeren Baumstruktur.
	TODO: Bild
	\\
	
	\item[\boldmath$4.$] Die Kanten, welche die Knotenpunkte verbinden, werden mithilfe von Zylindern visualisiert. 
	TODO: Bild
	\\
	

\end{description}


Was wird normalerweise noch vorgeschlagen, hier aber nicht verwirklicht?
Knoten-Dezimation
Curve Subdivision für sanftere Kurven
\section{Erweiterungen}

Vorgeschlagene Erweiterungen: Limitierungen der Abzweigungen, Limitierung der Knotentiefe, Wachstumsgeschwindigkeit in Abhängigkeit der Knotentiefe
Kurvenreduktion
