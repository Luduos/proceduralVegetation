\kurzfassung

In der folgenden Abschlussarbeit werden zwei verschiedene Verfahren zur prozeduralen Generierung von Baumstrukturen und die Implementierungen dieser innerhalb des Frameworks der Unreal Engine 4 vorgestellt. Basierend auf der Untersuchung bisheriger Arbeiten werden zwei Ansätze gewählt: Lindenmayer-Systeme und ein Space Colonization Algorithmus.

Ein Lindenmayer-System -- kurz: L-System -- ist eine Erweiterung von kontextfreien Grammatiken, welche Teile einer übergebenen Zeichenkette anhand festgelegter Regeln durch andere Zeichenketten ersetzen. \cite[S.2]{ABOP:04} Die grundlegende Funktionsweise und die für eine Generierung von Baumstrukturen benötigten Erweiterungen werden behandelt sowie eine Methode zur Visualisierung der Ergebnisse von L-Systemen vorgestellt.

Das Prinzip des verwendeten Space Colonization Algorithmus basiert auf einer biologisch motivierten Simulation der Konkurrenz von wachsenden Zweigen um Wachstumsraum. \cite[S.2f]{SpaceColonizationAlgorithm:07} Benötigte Eingaben, der Ablauf des Algorithmus und die Prozedur zur Generierung von Baumstrukturen werden erläutert sowie Erweiterungen des ursprünglichen Algorithmus vorgestellt.

Die im Rahmen dieser Arbeit umgesetzten Implementierungen beider Verfahren werden behandelt. Die verwendete Datenstruktur in Form eines graphentheoretischen Baumes und das darauf basierende Modellgenerierungssystem, welches für die Konstruktion der Modelldaten verantwortlich ist, werden vorgestellt. Die aus den Implementierungen resultierenden Baumstrukturen werden präsentiert und der Einfluss von Parametern auf das visuelle Erscheinungsbild sowie die Effizienz der Generierung der Modelle wird behandelt. 

Abschließend findet eine Bewertung und ein Vergleich beider Verfahren auf Grundlage der vorgestellten Ergebnisse sowie die Behandlung wünschenswerter Erweiterungen für zukünftige Arbeiten statt.\\
 
 In the following bachelor thesis we present two different methods for the procedural generation of tree-like structures and their implementations in the framework of the Unreal Engine 4. Based on the analysis of past works we chose two approaches: Lindenmayer-Systems and a Space Colonization Algorithm.
 
 A Lindenmayer-System -- in short: L-System -- is an extension of context free grammars which replace parts of a given string by other strings, according to a predefined ruleset. \cite[S.2]{ABOP:04} We examine the basic functionality and the extensions required for generating  tree-like structures and present the chosen visualization method for displaying the results.
 
 The functionality of the Space Colonization Algorithm is biologically motivated by the competition for space between growing branches. We introduce the necessary input for the algorithm, the algorithmic process, the procedure for generating tree-like structures and extensions of the original algorithm.
 
 We present the implementations of both techniques, the graph-theoretical tree data structure used by us and the system responsible for constructing the 3D model data. Afterwards we display the resulting tree structures and discuss the impact of parameter values on the visual appearance and efficiency of the generation methods.
 
 Finally we assess and compare both techniques based on the presented results and discuss desirable extensions reserved for future works.


